In this paper, we study the notion of $\epsilon$-Adverserial Stackelberg Value ($\mathbf{ASV}^{\epsilon}$) for two-player non-zero sum mean-payoff games, with the assumption that each player is almost rational. The $\mathbf{ASV}^{\epsilon}$ of Player~0 is the largest value that Player~0 can obtain when announcing her strategy to Player~1 who in turn responds with an $\epsilon$-best response. In this paper, we prove that the $\mathbf{ASV}^{\epsilon}$ is always achievable. We show how to compute this value, and prove that the threshold problem, i.e. $\mathbf{ASV}^{\epsilon} > c$, lies in $\textbf{NP}$ and can be achieved by a finite memory strategy. Furthermore, we study the effect on $\mathbf{ASV}^{\epsilon}$ when Player~0 is restricted to only finite memory strategies. This has to be compared with the same problems in the framework of Adverserial Stackelberg Values in general where the $\mathbf{ASV}$ is not always achievable.