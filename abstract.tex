Two-player Stackelberg games are non-zero sum strategic games comprising of a leader (Player~0) and a follower (Player~1). The leader announces her strategy, and the follower responds by playing an optimal response (or an $\epsilon$-optimal response) to maximise his own payoff. The follower, while maximising his own payoff, can either be co-operative, that is, he chooses a response that maximises the payoff of the leader, or he can be adversarial, thus choosing a response that also minimises the payoff of the leader. In this work, we study Stackelberg games for two-player non-zero sum mean-payoff setting where the follower is adversarial and $\epsilon$-optimal, for a fixed $\epsilon$.

The $\mathbf{ASV}^{\epsilon}$ of Player~0 is the supremum of the values that Player~0 can obtain by announcing her strategy to Player~1 who in turn responds with an $\epsilon$-optimal strategy.
We prove that the $\mathbf{ASV}^{\epsilon}$ is always achievable, possibly with an infinite memory strategy.
This is in contrast with the framework of two-player Adversarial Stackelberg mean-payoff games where the $\epsilon$ is not fixed.
We show that the threshold problem, i.e. given a rational constant $c$, deciding whether $\mathbf{ASV}^{\epsilon} > c$, is in $\textbf{NP}$, and a finite memory strategy for Player~0 suffices to achieve the threshold.
We also give an {\sf EXPTime} algorithm to compute the $\mathbf{ASV}^{\epsilon}$.
Furthermore, we study the effect on $\mathbf{ASV}^{\epsilon}$ when Player~0 is restricted to only finite memory strategies.
Finally, we improve upon some of the results obtained earlier in the framework of two-player Adversarial Stackelberg mean-payoff games where the $\epsilon$ is not fixed.
