Two-player Stackleberg games are strategic games comprising of a leader (Player~0) and a follower (Player~1). The leader announces her strategy, and the follower responds by playing a best responses (or an $\epsilon$-optimal best response). The follower can be either co-operative or adversarial. In this work, we study Stackelberg games for two-player non-zero sum mean-payoff setting where the follower is adversarial and $\epsilon$-optimal for a fixed $epsilon$.

% In this paper, we study the notion of $\epsilon$-Adversarial Stackleberg Value ($\mathbf{ASV}^{\epsilon}$) for two-player non-zero sum mean-payoff games, with the assumption that one player is almost rational and the other is completely rational.
The $\mathbf{ASV}^{\epsilon}$ of Player~0 is the largest value that Player 0 can obtain by announcing her strategy to Player~1 who in turn responds with an $\epsilon$-best response. 
We prove that the $\mathbf{ASV}^{\epsilon}$ is always achievable, possibly with an infinite memory strategy.
This is in contrast with the framework of two-player Adversarial Stackleberg mean-payoff games where the $epsilon$ is not fixed.
We show that the threshold problem, i.e. $\mathbf{ASV}^{\epsilon} > c$, lies in $\textbf{NP}$, and a finite memory strategy suffices for this decision problem.
We also give an {\sf EXPTime} algorithm to compute the $\mathbf{ASV}^{\epsilon}$ value.
Furthermore, we study the effect on $\mathbf{ASV}^{\epsilon}$ when Player~0 is restricted to only finite memory strategies.
Finally, we strengthen the results for some of the problems studied earlier in the framework of two-player Adversarial Stackleberg mean-payoff games where the $\epsilon$ is not fixed.