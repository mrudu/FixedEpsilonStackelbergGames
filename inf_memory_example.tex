\begin{theorem}
\label{ThmExNeedInfMem}
There exists a mean-payoff game $\mathcal{G}$ with vertex $v$ such that Player $0$ needs an infinite memory strategy to achieve the value $\mathbf{ASV}^{\epsilon}(v)$.
\end{theorem}
\begin{figure}
    \centering
    \begin{tikzpicture}[->,>=stealth',shorten >=1pt,auto,node distance=3.8cm,
                        semithick, squarednode/.style={rectangle, draw=red!60, fill=red!5, very thick, minimum size=15mm}]
      \tikzstyle{every state}=[fill=white,draw=black,text=black,minimum size=2cm]
    
      \node[squarednode]   (A)                    {$v_0$};
      \node[state]         (B) [left of=A] {$v_1$};
      \node[state]         (C) [right of=A] {$v_2$};
      \node[draw=none, fill=none, minimum size=0cm, node distance = 2cm]         (D) [below of=A] {$start$};
      \path (A) edge [bend left, below] node {(0,0)} (B)
                edge              node {(0,1)} (C)
                edge [loop above] node {(2,0)} (A)
            (B) edge [loop above] node {(0,2+2$\epsilon$)} (B)
                edge [bend left, above] node {(0,0)} (A)
            (C) edge [loop above] node {(0,1)} (C)
            (D) edge [left] node {} (A);
    \end{tikzpicture}
    \caption{No finite memory strategy $\sigma_0$ of Player $0$ to achieve $\mathbf{ASV}^{\epsilon}(v_0)$}
    \label{fig:no_finite_strategy}
\end{figure}
\begin{proof}
Consider the example in Figure \ref{fig:no_finite_strategy}. We show that $\mathbf{ASV}^{\epsilon}(v_0) = 1$ and that this $\mathbf{ASV}^{\epsilon}$ can only be achieved using an infinite memory strategy. Assume a strategy $\sigma_0$ for Player 0 defined as: starting with $k=1$, repeat forever, if Player 1 plays $v_0 \to v_0$ repeatedly at least $k$ times before playing $v_0 \to v_1$, then from $v_1$, play $v_1 \to v_1$ repeatedly $k$ times and then play $v_1 \to v_0$ and increment $k$ by 1, i.e., $k=k+1$; else, if Player 1 plays $v_0 \to v_0$ less than $k$ times before playing $v_0 \to v_1$, then from $v_1$ , play $v_1 \to v_0$.

The best response for Player 1 to strategy $\sigma_0$ would be to choose $k$ sequentially as $k = 1, 2, 3, \dotsc$, to get a play $\pi = ((v_0)^i(v_1)^i)_{i \in \mathbb{N}}$. We know that $\underline{\mathbf{MP}}_1(\pi) = 1+\epsilon$ and $\underline{\mathbf{MP}}_0(\pi) = 1$. Player 1 can sacrifice an amount that is less than $\epsilon$ to minimize mean-payoff of Player 0, and thus he would not like to play $v_0 \to v_2$. In particular, the strategy $\sigma_1$ of Player 1, which is defined as playing $v_0 \to v_2$, yields a mean-payoff of 1 for Player 1 and hence we conclude that $\sigma_1 \notin \mathbf{BR}_1^{\epsilon}(\sigma_0)$. Player 1 cannot play any other strategy without increasing the mean-payoff of Player 0 and/or decreasing his own payoff. Note that, it can be easily seen that Player 1 does not have a finite memory best-response strategy. Thus, the $\mathbf{ASV}^{\epsilon}(\sigma_0)(v_0) = 1$.


\mbcomment{We need to be more abstract about explaining the finite memory strategy}
We claim that $\mathbf{ASV}^{\epsilon}(\sigma_0)(v_0) = \mathbf{ASV}^{\epsilon}(v_0)$. For every strategy $\sigma_1$ of Player 1 such that $\sigma_1 \in \mathbf{BR}_1^{\epsilon}(\sigma_0)$, we note that the higher the payoff Player 1 has, the lower is the payoff for Player 0. For every other strategy, $\sigma_0'$, of Player 0, if best-response of Player 1 to $\sigma_0'$ gives a mean-payoff less than $1+\epsilon$, then Player 1 will switch to $v_2$, thus giving Player 0 a payoff of 0. If best-response of Player 1 to $\sigma_0'$ gives a mean-payoff greater than $1+\epsilon$, then Player 0 will have a lower $\mathbf{ASV}^{\epsilon}(\sigma_0')$. Towards achieving $\mathbf{ASV}^{\epsilon}$ value, we can define an ideal finite memory strategy $\sigma_0^{FM}$ of Player 0 which would be of the form: repeat forever, if Player 1 plays $v_0 \to v_0$ repeatedly at least $l$ times before playing $v_0 \to v_1$, then Player 0 plays $v_1 \to v_1$ repeatedly $k$ times before playing $v_1 \to v_0$, where $l, k$ are some constants in $\mathbb{N}$. The best response of Player 1 to this strategy would be to play: repeat forever, play $v_0 \to v_0$ exactly $l$ times before playing $v_0 \to v_1$. Note that the mean-payoff of the best response is $\frac{(2+2\epsilon)\cdot k}{k + l + 2}$ and it must be greater than $1 + \epsilon$ to ensure Player 1 does not switch to playing $v_0 \to v_2$. Thus we get that $k \geqslant l + 2$. Player 0 benefits the most if she chooses $k = l+2$. Thus, we get a slightly modified strategy $\sigma_0^{FM'}$ of Player 0: repeat forever, if Player 1 plays $v_0 \to v_0$ repeatedly at least $l$ times before playing $v_0 \to v_1$, then Player 0 plays $v_1 \to v_1$ repeatedly $l+2$ times before playing $v_1 \to v_0$, where $l$ is  some constant in $\mathbb{N}$. Player 1's best response here is to play: repeat forever, play $v_0 \to v_0$ exactly $l$ times before playing $v_0 \to v_1$. Note that every other $\epsilon$-best response of Player 1 increases the payoff of Player 0. Thus, we get $\mathbf{ASV}^{\epsilon}(\sigma_0^{FM'}) = \frac{l}{l+2}$. By increasing $l \to \infty$, we can get $\mathbf{ASV}^{\epsilon}(\sigma_0^{FM'})$ very close to 1, but reaching 1 without infinite memory would not be possible. Thus, we see that the $\mathbf{ASV}^{\epsilon}$ can be reached only via an infinite memory strategy.
\end{proof}