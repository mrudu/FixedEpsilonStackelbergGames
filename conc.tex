In this work, we have studied the notion of Adversarial Stackelberg Equilibria on two-player non-zero sum infinite duration mean-payoff games played on graph arenas. We also assumed that the follower is not completely rational, and hence would always choose an $\epsilon$-optimal best-response to the leader's strategy.

We show that infinite memory is required for the leader to achieve the $\mathbf{ASV}^{\epsilon}$. We also show that the follower may require infinite memory to play an $\epsilon$-optimal best-response. We show that the Threshold Problem, i.e, checking if $\mathbf{ASV}^{\epsilon} > c$ can be decided in $\mathsf{NP}$-time. Additionally, if $\mathbf{ASV}^{\epsilon} > c$, we can construct a finite memory strategy for the leader to achieve this threshold. We show that the $\mathbf{ASV}^{\epsilon}$ can be expressed as a formula in the theory of reals with addition. Furthermore, we establish that this formula can be expressed as a set of linear programs, which can be solved in $\mathsf{EXP}$-time to compute the $\mathbf{ASV}^{\epsilon}$. Finally, we show that $\mathbf{ASV}^{\epsilon}$ is always achievable.

For future work, we consider the following directions.
\begin{enumerate}
    \item \textbf{Multiple followers}: In this work, we studied stackelberg mean-payoff games with one adversarial $\epsilon$-optimal follower. One obvious line of research is to extend this work to sudy stackelberg mean-payoff games with multiple adversarial $\epsilon$-optimal followers. This would model how players would react in a more realistic situation.
    \item \textbf{Other Quantitative Objectives}: In this work, we studied stackelberg mean-payoff games where  each player has mean-payoff objectives. We can also extend this work to study stackelberg games with where the players have other quantitative objectives such as discounted sum and quantitative reachability.
    \item \textbf{Other Solution Concepts:} The main solution concept that was focused on in this work was Nash Equilibrium. Going forward, we could study other forms of solution concepts such as subgame perfect equilibrium in both the co-operative as well as adversarial setting.
    \item \textbf{Multiple leaders}: It would be interesting to explore a scenario where there are several leaders who are in an equilibrium (such as Nash), and each leader has several followers. Here, we can explore many lexicographically ordered objectives. For example, we could consider scenarios where the leaders are rational yet adversarial and thus, would try to maximize their own payoff first and then minimize the payoff of other leaders. 
\end{enumerate}