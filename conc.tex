In this work, we have studied the notion of Adversarial Stackelberg Equilibria on two-player non-zero sum infinite duration mean-payoff games played on graph arenas. We also assumed that the follower is not completely rational, and hence would always choose an $\epsilon$-optimal best-response to the leader's strategy.

We show that infinite memory is required for the leader to achieve the $\mathbf{ASV}^{\epsilon}$. We also show that the follower may require infinite memory to play an $\epsilon$-optimal best-response. We show that the Threshold Problem, i.e, checking if $\mathbf{ASV}^{\epsilon} > c$ can be decided in $\mathsf{NP}$-time. Additionally, if $\mathbf{ASV}^{\epsilon} > c$, we can construct a finite memory strategy for the leader to achieve this threshold.We show that the $\mathbf{ASV}^{\epsilon}$ can be expressed as a formula in the theory of reals with addition. Furthermore, we establish that this formula can be expressed as a set of linear programs, which can be solved in $\mathsf{EXP}$-time to compute the $\mathbf{ASV}^{\epsilon}$.Finally, we show that $\mathbf{ASV}^{\epsilon}$ is always achievable.

For future work, we consider the following directions.
\begin{enumerate}
    \item The case of multiple followers
    \item In this work, we considered the case where each player has mean-payoff objectives. This can be extended to other kinds of quantitative objectives like discounted sum, quantitative reachability for ASV-$\epsilon$.
    \item other forms of solution concepts like subgame perfect equilibrium instead of Nash equilibrium as has been considered here; one can also study both CSV and ASV for SPEs.
    \item A more realistic scenario would be to consider several leaders who are in an equilibrium (like Nash), and each leader has several followers.
\end{enumerate}