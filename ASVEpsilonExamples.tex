In this \mychapter, we start the study of mean-payoff games by displaying some constraints on the memory required to achieve $\mathbf{ASV}^{\epsilon}$.

First we show that there exist mean-payoff games $\mathcal{G}$ and vertex $v_0$ in $\mathcal{G}$ such that $\mathbf{ASV}^{\epsilon}(v_0)$ can be achieved using infinite memory strategy for Player $0$, and also there exist mean-payoff games $\mathcal{G}$ with vertex $v_0$ such that $\mathbf{ASV}^{\epsilon}(v_0)$ can be achieved using finite memory (but not memoryless) strategy for Player $0$.

\begin{theorem}
\label{ThmExNeedInfMem}
There exists a mean-payoff game $\mathcal{G}$ with vertex $v$ such that Player $0$ needs an infinite memory strategy to achieve
% the value
$\mathbf{ASV}^{\epsilon}(v)$.
\end{theorem}
\begin{figure}
    \centering
    \begin{tikzpicture}[->,>=stealth',shorten >=1pt,auto,node distance=3.8cm,
                        semithick, squarednode/.style={rectangle, draw=blue!60, fill=blue!5, very thick, minimum size=15mm}]
      \tikzstyle{every state}=[fill=white,draw=black,text=black,minimum size=2cm]
    
      \node[squarednode]   (A)                    {$v_0$};
      \node[state, draw=red!60, fill=red!5]         (B) [left of=A] {$v_1$};
      \node[state, draw=red!60, fill=red!5]         (C) [right of=A] {$v_2$};
      \node[draw=none, fill=none, minimum size=0cm, node distance = 2cm]         (D) [below of=A] {$start$};
      \path (A) edge [bend left, below] node {(0,0)} (B)
                edge              node {(0,1)} (C)
                edge [loop above] node {(2,0)} (A)
            (B) edge [loop above] node {(0,2+2$\epsilon$)} (B)
                edge [bend left, above] node {(0,0)} (A)
            (C) edge [loop above] node {(0,1)} (C)
            (D) edge [left] node {} (A);
    \end{tikzpicture}
    \caption{Finite memory strategy of Player~$0$ may not achieve $\mathbf{ASV}^{\epsilon}(v_0)$}
    \label{fig:no_finite_strategy}
\end{figure}
\begin{proof}
Consider the example in Figure \ref{fig:no_finite_strategy}. We show that in this example $\mathbf{ASV}^{\epsilon}(v_0) = 1$, and that this $\mathbf{ASV}^{\epsilon}$ can only be achieved using an infinite memory strategy. Assume a strategy $\sigma_0$ for Player~0
such that the game is played in rounds.
In round $k$
% defined as: starting with $k=1$, repeat forever, 
\begin{itemize}[-]
    \item if Player~1 plays $v_0 \to v_0$ repeatedly at least $k$ times before playing $v_0 \to v_1$, then from $v_1$, play $v_1 \to v_1$ repeatedly $k$ times and then play $v_1 \to v_0$ and 
    % increment $k$ by 1, i.e., $k=k+1$; 
    move to round $k+1$;
    \item else, if Player~1 plays $v_0 \to v_0$ less than $k$ times before playing $v_0 \to v_1$, then from $v_1$ , play $v_1 \to v_0$.
\end{itemize}

The best response for Player~1 to strategy $\sigma_0$ would be to choose $k$ sequentially as $k = 1, 2, 3, \dotsc$, to get a play $\pi = ((v_0)^i(v_1)^i)_{i \in \mathbb{N}}$. We know that $\underline{\mathbf{MP}}_1(\pi) = 1+\epsilon$ and $\underline{\mathbf{MP}}_0(\pi) = 1$. Player~1 can sacrifice an amount that is less than $\epsilon$ to minimize mean-payoff of Player~0, and thus he would not like to play $v_0 \to v_2$. In particular, a strategy $\sigma_1$ of Player~1 that prescribes playing the edge $v_0 \to v_2$ some time 
% which is defined as playing $v_0 \to v_2$,
yields a mean-payoff of 1 for Player~1 and hence we conclude that $\sigma_1 \notin \mathbf{BR}_1^{\epsilon}(\sigma_0)$. Player~1 cannot play any other strategy without increasing the mean-payoff of Player~0 and/or decreasing his own payoff.
% Note that, it can be easily seen 
We can see that Player~1 does not have a finite memory best-response strategy. Thus, the $\mathbf{ASV}^{\epsilon}(\sigma_0)(v_0) = 1$.

% \mbcomment{We need to be more abstract about explaining the finite memory strategy}
We claim that $\mathbf{ASV}^{\epsilon}(\sigma_0)(v_0) = \mathbf{ASV}^{\epsilon}(v_0)$. For every strategy $\sigma_1$ of Player~1 such that $\sigma_1 \in \mathbf{BR}_1^{\epsilon}(\sigma_0)$, we note that the higher the payoff Player~1 has, the lower is the payoff for Player~0. For every other strategy, $\sigma_0'$, of Player~0, if best-response of Player~1 to $\sigma_0'$ gives a mean-payoff less than $1+\epsilon$, then Player~1 will switch to $v_2$, thus giving Player~0 a payoff of 0. If best-response of Player~1 to $\sigma_0'$ gives a mean-payoff greater than $1+\epsilon$, then Player~0 will have a lower $\mathbf{ASV}^{\epsilon}(\sigma_0')$.

Consider a finite memory strategy of Player~0. If Player~1 has an infinite memory response (that cannot be encoded by finite memory), it can only lead to looping over $v_0$ more and more, this gives him a payoff
which is eventually 0.

Now consider a finite memory response of Player~1 to the finite memory strategy of Player~0. Note that the resultant outcome is a regular path. Note that Player~0 would choose a finite memory strategy such that the
best response of Player~1 (which is achievable) gives him a value of at least $1 + \epsilon$. Also since both players have finite memory strategies, the resultant outcome is a regular path over vertices $v_0$ and $v_1$. In
every such regular path, the effect of the edge from $v_0$ to $v_1$ and the
edge from $v_1$ to $v_0$ is non-negligible, and hence if the Player~1 is at
least $1 + \epsilon$, the payoff of Player~0 will be less than 1. Thus no
finite memory strategy can achieve an $\mathbf{ASV}^{\epsilon}$ that is equal to 1. 
% Towards achieving $\mathbf{ASV}^{\epsilon}$, we can define an ideal finite memory strategy $\sigma_0^{FM}$ of Player~0 which would be of the form: repeat forever, if Player~1 plays $v_0 \to v_0$ repeatedly at least $l$ times before playing $v_0 \to v_1$, then Player~0 plays $v_1 \to v_1$ repeatedly $k$ times before playing $v_1 \to v_0$, where $l, k$ are some constants in $\mathbb{N}$. The best response of Player~1 to this strategy would be to play: repeat forever, play $v_0 \to v_0$ exactly $l$ times before playing $v_0 \to v_1$. Note that the mean-payoff of the best response is $\frac{(2+2\epsilon)\cdot k}{k + l + 2}$ and it must be greater than $1 + \epsilon$ to ensure Player~1 does not switch to playing $v_0 \to v_2$. Thus we get that $k \geqslant l + 2$. Player~0 benefits the most if she chooses $k = l+2$. Thus, we get a slightly modified strategy $\sigma_0^{FM'}$ of Player~0: repeat forever, if Player~1 plays $v_0 \to v_0$ repeatedly at least $l$ times before playing $v_0 \to v_1$, then Player~0 plays $v_1 \to v_1$ repeatedly $l+2$ times before playing $v_1 \to v_0$, where $l$ is  some constant in $\mathbb{N}$. Player~1's best response here is to play: repeat forever, play $v_0 \to v_0$ exactly $l$ times before playing $v_0 \to v_1$. Note that every other $\epsilon$-best response of Player~1 increases the payoff of Player~0. Thus, we get $\mathbf{ASV}^{\epsilon}(\sigma_0^{FM'}) = \frac{l}{l+2}$. By increasing $l \to \infty$, we can get $\mathbf{ASV}^{\epsilon}(\sigma_0^{FM'})$ very close to 1, but reaching 1 without infinite memory would not be possible. Thus, we see that the $\mathbf{ASV}^{\epsilon}$ can be reached only via an infinite memory strategy.
\end{proof}

The following example shows the existence of mean-payoff games in which Player $0$ can achieve the adversarial value with finite memory (but not memoryless) strategies.

\begin{theorem}
\label{ThmExNeedFinMem}
There exists a mean-payoff game $\mathcal{G}$ with vertex $v$ such that a finite memory of Player $0$ suffices to achieve $\mathbf{ASV}^{\epsilon}(v)$.
\end{theorem}
\begin{figure}
    \centering
    \begin{tikzpicture}[->,>=stealth',shorten >=1pt,auto,node distance=3.8cm,
                        semithick, squarednode/.style={rectangle, draw=red!60, fill=red!5, very thick, minimum size=15mm}]
      \tikzstyle{every state}=[fill=white,draw=black,text=black,minimum size=2cm]
    
      \node[squarednode]   (A)                    {$v_0$};
      \node[state]         (B) [left of=A] {$v_1$};
      \node[state]         (C) [right of=A] {$v_2$};
      \node[draw=none, fill=none, minimum size=0cm, node distance = 2cm]         (D) [above of=A] {$start$};
      \path (A) edge [bend left, below] node {(1,1)} (B)
                edge              node {(0,1)} (C)
            (B) edge [loop above] node {(0,2)} (B)
                edge [bend left, above] node {(1,1)} (A)
            (C) edge [loop above] node {(0,1)} (C)
            (D) edge [left] node {} (A);
    \end{tikzpicture}
    \caption{An example in which a finite memory strategy for Player $0$ suffices to achieve $\mathbf{ASV}^{\epsilon}(v_0)$.}
    \label{fig:finite_strategy_response}
\end{figure}
\begin{proof}
Consider the example in Figure \ref{fig:finite_strategy_response}. We show that $\mathbf{ASV}^{\epsilon}(v_0) = 1 - \epsilon$. Assume a strategy $\sigma_0^{k}$ for Player 0 defined as: repeat forever, from $v_1$ play one time $v_1 \to v_1$ and then repeat playing $v_1 \to v_0$ for $k$ times, with $k$ chosen such that mean-payoff for Player 0 is equal to $1 - \epsilon$. Such a $k$ always exists. The $\epsilon$-best response of Player 1 to $\sigma_0^{k}$ is to always play $v_0 \to v_1$, as by playing this edge forever, Player 1 gets a mean-payoff equal to $1+\epsilon$, where as if Player 1 plays $v_0 \to v_2$, then Player 1 is receives a payoff of $1$. Since $1 \ngtr (1 + \epsilon) - \epsilon$, the strategy $v_0 \to v_2$ does not fall under the $\epsilon$-best-response for Player 1, thus forcing Player 1 to play $v_0 \to v_1$. Thus $\mathbf{ASV}^{\epsilon}(v_0)$ is achieved with a finite memory of size $k$ for Player 0. Note that this size $k$ is a function of $\epsilon$.
\end{proof}

Further, we can also show that for a given strategy $\sigma_0$ of Player $0$, and for a given $\epsilon > 0$, there may not exist a finite memory strategy of Player $1$ that is an $\epsilon$-optimal response to $\sigma_0$.

\begin{theorem}
\label{ThmP1NeedInfMem}
There exists a mean-payoff game $\mathcal{G}$ such that for some Player $0$ strategy $\sigma_0$, for every strategy $\sigma_1$ of Player $1$, where $\sigma_1 \in \mathbf{BR}_1^{\epsilon}(\sigma_0)$, Player 1 needs infinite memory to play $\sigma_1$.
\end{theorem}
\begin{figure}
    \centering
    \begin{tikzpicture}[->,>=stealth',shorten >=1pt,auto,node distance=3.8cm,
                        semithick, squarednode/.style={rectangle, draw=red!60, fill=red!5, very thick, minimum size=15mm}]
      \tikzstyle{every state}=[fill=white,draw=black,text=black,minimum size=2cm]
    
      \node[state]   (A)                    {$v_0$};
      \node[squarednode]         (B) [left of=A] {$v_1$};
      \node[state]         (C) [right of=A] {$v_2$};
      \node[draw=none, fill=none, minimum size=0cm, node distance = 2cm]         (D) [below of=A] {$start$};
      \path (A) edge [bend left, below] node {(0,0)} (B)
                edge              node {(0,0)} (C)
                edge [loop above] node {(0,3)} (A)
            (B) edge [loop above] node {(3,0)} (B)
                edge [bend left, above] node {(0,0)} (A)
            (C) edge [loop above] node {(1,0)} (C)
            (D) edge [left] node {} (A);
    \end{tikzpicture}
    \caption{No $\epsilon$-optimal finite memory response of Player $1$ to a strategy $\sigma_0$ of Player $0$}
    \label{fig:no_optimal_response}
\end{figure}
\begin{proof}
Consider the example in Figure \ref{fig:no_optimal_response}.
Consider the following strategy $\sigma_0$ of Player $0$.
Player $0$ loops over $v_0$ $i$ times, and then sends the token to $v_1$.
Player $1$ loops over $v_1$ $k$ times, and then sends the token to $v_0$.
If $k \geqslant i$, then Player $0$ increases $i$ by $1$, and repeats the above, otherwise sends the token to $v_2$.
Clearly for all $\epsilon \leqslant 1.5$, no finite memory strategy only is an $\epsilon$ best response to $\sigma_0$.
\end{proof}