In this \mychapter, we study the threshold problem of determining if $\mathbf{ASV}^{\epsilon}(v) > c$  for a given game $\mathcal{G}$ where $v$ is a vertex in $\mathcal{G}$ and $c$ is some rational constant.

We start by showing that if $\mathbf{ASV}^{\epsilon}(v) > c$, then there exists a strategy $\sigma_0$ for Player~0 that enforces $\mathbf{ASV}^{\epsilon}(\sigma_0)(v) > c$.

\begin{lemma}
\label{lem:witness_strategy}
For all mean-payoff games $\mathcal{G}$, for all vertices $v \in V$, for a fixed $\epsilon > 0$, we have that $\mathsf{ASV}^{\epsilon}(v) > c$ iff there exists a strategy $\sigma_0 \in \Sigma_0$ such that $\mathsf{ASV}^{\epsilon}(\sigma_0)(v) > c$.
\end{lemma}
\begin{proof}
% By definition of $\mathsf{ASV}^{\epsilon}(\sigma_0)(v) > c$, we have to show that:
% \begin{equation*}
%     \mathsf{ASV}^{\epsilon}(v) > c \iff \exists \sigma_0: \mathbf{BR}^{\epsilon}_1(\sigma_0) \neq \varnothing \land \forall \sigma_1 \in \mathbf{BR}^{\epsilon}_1(\sigma_0): \underline{\mathbf{MP}}_0(\mathbf{Out}_v(\sigma_0,\sigma_1)) > c
% \end{equation*}

\noindent The right to left direction of the proof is trivial as $\sigma_0$ can play the role of witness for $\mathsf{ASV}^{\epsilon}(v) > c$, i.e., if there exists a strategy $\sigma_0$ of Player $0$ such that $\mathsf{ASV}^{\epsilon}(\sigma_0)(v) > c$, then $\mathsf{ASV}^{\epsilon}(v) > c$.
% \\
% \noindent 

For the left to right direction of the proof, let $\mathsf{ASV}^{\epsilon}(v) = c'$. By definition of $\mathsf{ASV}^{\epsilon}( v)$, we have:
\begin{equation*}
    c' = \sup\limits_{\sigma_0 \in \Sigma_0}\mathsf{ASV}^{\epsilon}(\sigma_0)(v) 
\end{equation*}

\noindent By definition of $\sup$, for all $\delta > 0$, we have that:
\begin{equation*}
    \exists \sigma_0^{\delta}: \mathsf{ASV}^{\epsilon}(\sigma_0^{\delta})(v) \geqslant c' - \delta
\end{equation*}

\noindent Let us consider a $\delta > 0$ such that $c' - \delta > c$. Such a $\delta$ exists as $c' > c$. Then we obtain:
\begin{equation*}
    \exists \sigma_0: \mathsf{ASV}^{\epsilon}(\sigma_0)(v) \geqslant c' - \delta > c
\end{equation*}
\end{proof}

\noindent\textbf{Witnesses for $\mathbf{ASV}^{\epsilon}$} For a mean-payoff game $\mathcal{G}$ and an $\epsilon > 0$, we associate with each vertex $v$ in $\mathcal{G}$, the following set of pairs of real numbers:
\begin{equation*}
\Lambda^{\epsilon}(v) = \{(c,d) \in \mathbb{R}^2 \mid v \models \ll 1 \gg \underline{\mathbf{MP}}_0(\pi) \leqslant c \land \underline{\mathbf{MP}}_1(\pi) > d-\epsilon \}
\end{equation*}
We say that a vertex $v$ is $(c,d)^{\epsilon}$-bad if $(c,d) \in \Lambda^{\epsilon}(v)$. Let $c' \in \mathbb{R}$. A play $\pi$ in G is called a $(c',d)^{\epsilon}$-witness of $\mathbf{ASV}^{\epsilon}(v) > c$ if it starts from $v$, and $(\underline{\mathbf{MP}}_0(\pi), \underline{\mathbf{MP}}_1(\pi)) = (c', d)$ where $c' > c$, and $\pi$ does not contain any $(c,d)^{\epsilon}$-bad vertex. A play $\pi$ is called an $\epsilon$-witness of $\mathbf{ASV}^{\epsilon}(v) > c$ if it is a $(c',d)^{\epsilon}$-witness of $\mathbf{ASV}^{\epsilon}(v) > c$ for some $c',d$. 
The following theorem justifies the name witness.

\sgcomment{We should state the difficulty in our case compared to \cite{FGR20}.}
A similar result has been established in \cite{FGR20} where the authors show that $\mathbf{ASV}(v) > c$ if and only if there exists a witness for $\mathbf{ASV}(v) > c$. However, proving that if $\mathbf{ASV}(v) > c$, then there exists an $\epsilon$-witness for $\mathbf{ASV}(v) > c$ requires a different proof technique, and is much more challenging in our case.

\begin{theorem}
\label{ThmWitnessASVInfMem}
For a given mean-payoff game $\mathcal{G}$ with a set $V$ of vertices, and for a vertex $v \in V$, we have that $\mathbf{ASV}^{\epsilon}(v) > c$ if and only if there exists a 
$(c',d)^{\epsilon}$-witness of $\mathbf{ASV}^{\epsilon}(v) > c$.
\end{theorem}
\begin{proof}
First we prove the right to left direction, i.e., we are given a play $\pi$ in $\mathcal{G}$ that starts from $v$   and the path $\pi$ is such that $(\underline{\mathbf{MP}}_0(\pi), \underline{\mathbf{MP}}_1(\pi)) = (c', d)$ for $c' > c$ and does not cross a $(c,d)^{\epsilon}$-bad vertex. We need to prove that $\mathbf{ASV}^{\epsilon}(v) > c$.
\\
\\
We do this by defining a strategy $\sigma_0$ for Player~0, such that $\mathbf{ASV}^{\epsilon}(\sigma_0)(v) > c$:
\begin{enumerate}
    \item $\forall h \leqslant \pi$, and $last(h)$ is a Player~0 vertex, the strategy $\sigma_0$ is such that $\sigma_0(h)$ follows $\pi$.
    \item $\forall h \nleqslant \pi$, where there has been a deviation from $\pi$ by Player~1, we assume Player~0 switches to a strategy that we call \textit{punishing}. This strategy is defined as follows: In the subgame after history $h'$ where ${\sf last}(h')$ is the first vertex from which Player~1 deviate from $\pi$, we know that Player~0 has a strategy to enforce the objective: $\underline{\mathbf{MP}}_0(\pi) > c$ $\lor$ $ \underline{\mathbf{MP}}_1(\pi) \leqslant d-\epsilon$. This is true because $\pi$ does not cross any $(c,d)^{\epsilon}$-bad vertex and since $n$-dimensional mean-payoff games are determined.
\end{enumerate}

Let us now establish that the strategy $\sigma_0$ satisfies $\mathbf{ASV}^{\epsilon}(\sigma_0)(v) > c$. 
First note that, since $\underline{\mathbf{MP}}_1(\pi) = d$, we have that $\sup\limits_{\sigma_1 \in \mathbf{BR}_1^{\epsilon}(\sigma_0)} \underline{\mathbf{MP}}_1(\mathbf{Out}(\sigma_0, \sigma_1)) \geqslant d$. Now consider some strategy $\sigma_1' \in \mathbf{BR}_1^{\epsilon}(\sigma_0)$ and let $\pi' = \mathbf{Out}(\sigma_0, \sigma_1')$. Clearly, $\pi'$ is such that $\underline{\mathbf{MP}}_1(\pi') > \sup\limits_{\sigma_1 \in \mathbf{BR}_1^{\epsilon}(\sigma_0)} \underline{\mathbf{MP}}_1(\mathbf{Out}(\sigma_0, \sigma_1)) -\epsilon \geqslant d-\epsilon$. If $\pi' = \pi$, we know that $\underline{\mathbf{MP}}_0(\pi') > c$. If $\pi' \neq \pi$, then when $\pi'$ deviates from $\pi$ we know that Player~0 employs a punishing strategy, thus making sure that $\underline{\mathbf{MP}}_0(\pi') > c$ $\lor$ $ \underline{\mathbf{MP}}_1(\pi') \leqslant d-\epsilon$. Since $\sigma_1' \in \mathbf{BR}_1^{\epsilon}(\sigma_0)$, it must be true that $\underline{\mathbf{MP}}_0(\pi') > c$. Thus, $\forall \sigma_1' \in \mathbf{BR}_1^{\epsilon}(\sigma_0)$, we have $\underline{\mathbf{MP}}_0(\mathbf{Out}(\sigma_0, \sigma_1')) > c$. Therefore, $\mathbf{ASV}^{\epsilon}(\sigma_0)(v) > c$, which implies $\mathbf{ASV}^{\epsilon}(v) > c$.

Now we consider the left to right direction of the proof, i.e., we are given that $\mathbf{ASV}^{\epsilon}(v) > c$. 
Hence by \textbf{\cref{lem:witness_strategy}}, there exists a strategy $\sigma_0$ for Player $0$ such that $\mathbf{ASV}^{\epsilon}(\sigma_0)(v) > c$. Thus, for some $\delta > 0$, we have:
\begin{equation*}
    \inf\limits_{\sigma_1 \in \mathbf{BR}_1^{\epsilon}(\sigma_0)} \underline{\mathbf{MP}}_0(\mathbf{Out}(\sigma_0, \sigma_1)) = c' = c + \delta
\end{equation*}
Let $d = \sup\limits_{\sigma_1 \in \mathbf{BR}_1^{\epsilon}(\sigma_0)} \underline{\mathbf{MP}}_1(\mathbf{Out}(\sigma_0, \sigma_1))$. We first prove that for all $\sigma_1 \in \mathbf{BR}_1^{\epsilon}(\sigma_0)$, we have that $\mathbf{Out}_v(\sigma_0, \sigma_1)$ does not cross a $(c,d)^{\epsilon}$-bad vertex. For every $\sigma_1 \in \mathbf{BR}_1^{\epsilon}(\sigma_0)$, we let $\pi_{\sigma_1} = \mathbf{Out}_v(\sigma_0, \sigma_1)$. We note that $\underline{\mathbf{MP}}_1(\pi_{\sigma_1}) > d -\epsilon$ and $\underline{\mathbf{MP}}_0(\pi_{\sigma_1}) > c$. For every $\pi' \in \mathbf{Out}_v(\sigma_0)$, we know that if $\underline{\mathbf{MP}}_1(\pi') > d - \epsilon$, then there exists a strategy $\sigma_1' \in \mathbf{BR}_1^{\epsilon}(\sigma_0)$ such that $\pi' = \mathbf{Out}_v(\sigma_0, \sigma_1')$. This means that $\underline{\mathbf{MP}}_0(\pi') > c$. Thus we can see that every deviation from $\pi_{\sigma_1}$ either gives Player~1 a mean-payoff less than $d - \epsilon$ or Player~0 a mean-payoff greater than $c$. Therefore, we conclude that $\pi_{\sigma_1}$ does not cross any $(c,d)^{\epsilon}$-bad vertex. 

Now consider a sequence ($\sigma_i$)$_{i \in \mathbb{N}}$ of Player $1$ strategies such that $\sigma_i \in \mathbf{BR}_1^{\epsilon}(\sigma_0)$ for all $i \in \mathbb{N}$, and $\lim \limits_{i \to \infty} \underline{\mathbf{MP}}_1(\mathbf{Out}(\sigma_0, \sigma_i))=d$. Let $\pi_i = \mathbf{Out}(\sigma_0, \sigma_i)$.
Let $\inf(\pi_i)$ be the set of vertices that occur infinitely often in $\pi_i$, and let $V_{\pi_i}$ be the set of vertices appearing along the path $\pi_i$.
W.l.o.g., since there are finitely many SCCs, we can assume that let for all $i, j \in \mathbb{N}$, we have that $\inf(\pi_i) = \inf(\pi_j)$, that is, all the paths end up in the same SCC, say $S$, and also $V_{\pi_i} = V_{\pi_j}=V_{\pi}$ (say). 
Note that $S \subseteq V_{\pi}$.
For all vertices $v$ in the SCC $S$, we know that $v$ is not $(c'',d)^{\epsilon}$-bad for $c'\geqslant c''> c$.

Note that for every $\epsilon \ge \delta > 0$, there is a strategy $\sigma_1^\delta \in \mathbf{BR}_1^{\epsilon}(\sigma_0)$ of Player $1$, and a corresponding path $\pi' = \mathbf{Out}_v(\sigma_0, \sigma_1^\delta)$ such that $\underline{\mathbf{MP}}_1(\pi') > d-\delta$, and $\underline{\mathbf{MP}}_0(\pi') \geqslant c''$.
Also the set $V_{\pi'}$ of vertices appearing in $\pi'$ be such that $V_{\pi'} \subseteq V_\pi$, and $\inf(\pi') \subseteq S$.

Now since $F_{\mathsf{min}}(\mathbb{CH}(\mathbf{C}(S)))$ is a closed set, we have that $(c'',d) \in F_{\mathsf{min}}(\mathbb{CH}(\mathbf{C}(S)))$.
By \textbf{\cref{lemCHToPlay}}, there exists a play $\pi^*$ such that $(\underline{\mathbf{MP}}_0(\pi^*), \underline{\mathbf{MP}}_1(\pi^*)) = (c'',d)$.
Also $\inf(\pi^*) \subseteq S$, and $V_{\pi^*} \subseteq V_{\pi}$.
The proof follows since for all vertices $v \in V_\pi$, we have that $v$ is not $(c'',d)^{\epsilon}$-bad for $c'' > c$.
% $\inf(\pi^*) \subseteq S$, and $V_{\pi^*} \subseteq V_{\pi}$, and $(\underline{\mathbf{MP}}_0(\pi^*), \underline{\mathbf{MP}}_1(\pi^*)) = (c^*,d)$. We observe that $\pi^*$ does not cross a $(c,d)^{\epsilon}$-bad vertex since $V_{\pi^*} \subseteq V_{\pi}$, and thus is an $\epsilon$-witness to $\mathbf{ASV}^{\epsilon}(v) > c$.
\end{proof}

\begin{figure}
    \centering
    \begin{tikzpicture}
        \tkzDefPoint(0,0){O}
        \tkzDefPoint(8,0){P}
        \tkzDefPoint(8,5){Q}
        \tkzDefPoint(0,5){R}
        \tkzDefPoint(2.5,0){X}
        \tkzDefPoint(1.25,0){X'}
        \tkzDefPoint(8,3){Y}
        \tkzDefPoint(2.5,5){Z}
        \tkzDefPoint(0,3){W}
        \tkzDrawSegments(O,P P,Q Q,R R,O X,Z Y,W)
        \tkzDefPoint(5.5,0){A}
        \tkzDefPoint(2.5,3){B}
        \tkzDefPoint(1.25,3){B'}
        \tkzCalcLength(X,A)\tkzGetLength{radius}
        \tkzDrawArc[R with nodes, color=blue](X,\radius pt)(A,B)
        \tkzDrawArc[R with nodes, color=red](X',\radius pt)(A,B)
        \tkzDrawArc[R with nodes, color=red, style=dashed](X',\radius pt)(A,B')
        \tkzDrawPoints(B, B')
        \tkzLabelPoint[above](B'){$(< c', d)$}
        \tkzLabelPoint[above](B){$(c', d)$}
        \tkzLabelPoint[above](Z){$x = c'$}
        \tkzLabelPoint[right](Y){$y = d$}
    \end{tikzpicture}
    \caption{Plot of mean-payoffs when approaching the limit}
    \label{fig:plot_mp_limit}
\end{figure}

% Since $F_{\mathsf{min}}(\mathbb{CH}(\mathbf{C}(S)))$ is a closed set, we have that $(c^*, d) \in \mathbf{CH(\mathbb{C}(\emph{S}))}$ for some $c^* \in \mathbb{Q}$. We now show that $c^* \geqslant c' = c+\delta$ and at the limit when $i\to \infty$, we have that mean-payoff of Player~1 in $\pi_i$ approaches $d$ and mean-payoff of Player~0 in $\pi_i$ is greater than or equal to $c'$. We prove this by contradiction. 
\mbcomment{Might need to improve this.}
To give some intuition about the left to right direction of the proof, we consider the plots in Figure \ref{fig:plot_mp_limit}. The $x$-coordinates represent Player~0's mean-payoff, the $y$-coordinates represent Player~1's mean-payoff. The blue line depicts the set of plays $\pi_i$ such that when mean-payoff of Player~1 approaches $d$, mean-payoff of Player~0 approaches $c^* = c'$. Assume for contradiction that $c^* < c'$, i.e., we consider a set of outcomes $\pi_i$, where $\pi_i=\mathbf{Out}(\sigma_0, \sigma_i)$ and $\sigma_i \in \mathbf{BR}_1^{\epsilon}(\sigma_0)$, such that when mean-payoff of Player~1 approaches $d$, mean-payoff of Player~0 approaches $c^* < c'$. The red line on the graph plots such a set of $\pi_i$s. We note that, in the second case, that there exists outcomes ,$\pi_i$, as represented on graph by the red dashes, where $\underline{\mathbf{MP}}_0(\pi_i) < c'$ and $\underline{\mathbf{MP}}_1(\pi_i) < d$. This is a contradiction as the $\inf\limits_{\sigma_1 \in \mathbf{BR}_1^{\epsilon}(\sigma_0)} \underline{\mathbf{MP}}_0(\mathbf{Out}(\sigma_0, \sigma_1)) = c' = c + \delta$. Thus, it has to be the case that $c^* \geqslant c'$. 

Now, we establish \textbf{NP}-membership for the threshold problem of determining if $\mathbf{ASV}^{\epsilon}(v) > c$. Towards this, we first prove the following lemma.

\begin{lemma}
\label{LemPlaysAsWitnessForASV}
Given a mean-payoff game $\mathcal{G}$, a vertex $v$, an $\epsilon > 0$ and $c \in \mathbb{Q}$, we have that  $\mathbf{ASV}^{\epsilon}(v) > c$ if and only if there exist three acyclic plays $\pi_1, \pi_2, \pi_3$, and two simple cycles $l_1 , l_2$ such that:
\begin{enumerate}
    \item \textit{first}($\pi_1$) $ = v$, \textit{first}($\pi_2$) = \textit{last}($\pi_1$), \textit{first}($\pi_3$) = \textit{last}($\pi_2$), \textit{first}($\pi_2$) = \textit{last}($\pi_3$), and \textit{first}($\pi_2$) = \textit{first}($l_1$), and \textit{first}($\pi_3$) = \textit{first}($l_2$).
    \item there exists $\alpha, \beta \in \mathbb{Q}^{+}$, where $\alpha + \beta = 1$, such that:
    \begin{enumerate}
        \item $\alpha \cdot w_0(l_1) + \beta \cdot w_0(l_2) = c' > c$
        \item $\alpha \cdot w_1(l_1) + \beta \cdot w_1(l_2) = d$
    \end{enumerate}
    \item there is no $(c,d)^{\epsilon}$-bad vertex $v'$ along $\pi_1, \pi_2, \pi_3, l_1$ and $l_2$.
\end{enumerate}
\end{lemma}
\begin{proof}
This proof is similar to the proof of \textbf{Lemma 10} in \cite{FGR20}. 
For the right to left direction of the proof, where we are given finite acyclic plays $\pi_1, \pi_2, \pi_3$, simple cycles $l_1$ and $l_2$ and constants $\alpha, \beta$, we consider the witness $\pi = \pi_1\rho_1\rho_2\rho_3\dots$ where, for all $i \in \mathbb{N}$, we let $\rho_i = l_1^{[\alpha i]}.\pi_2.l_2^{[\beta i]}.\pi_3$. 
We know that $\underline{\mathbf{MP}}_1(\pi) = \alpha \cdot w_1(l_1) + \beta \cdot w_1(l_2) = d$ and $\underline{\mathbf{MP}}_0(\pi) = \alpha \cdot w_0(l_1) + \beta \cdot w_0(l_2) > c$. For all vertices $v$ in $\pi_1, \pi_2, \pi_3, l_1$ and $l_2$, it is given that $v$ is not $(c,d)^\epsilon$-bad. 
Therefore, $\pi$ is a suitable $\epsilon$-witness thus proving from \textbf{\cref{ThmWitnessASVInfMem}} that $\mathbf{ASV}^{\epsilon}(v) > c$.

We can optimize the $\epsilon$-witness $\pi$ by constructing an $\epsilon$-regular-witness for $\mathbf{ASV}^{\epsilon}(v) > c$ by modifying $\pi$ as follows:
\begin{caseof}
    \case{$w_0(l_1) > w_0(l_2)$ and $w_1(l_1) < w_1(l_2)$}
    Here, one simple cycle, $l_1$, increases Player 0's mean-payoff while the other simple cycle, $l_2$, increases Player 1's mean-payoff. We can build a witness $\pi' = \pi_1.(l_1^{[\alpha]k}.\pi_2.l_2^{[\beta+\tau]k}.\pi_3)^{\omega}$ for some very large $k \in \mathbb{N}$ and for some very small $\tau > 0$ such that $\underline{\mathbf{MP}}_0(\pi') > c$ and $\underline{\mathbf{MP}}_1(\pi') = d$.\footnote{For more details, we refer the reader to the Appendix.}
    \case{$w_0(l_1) < w_0(l_2)$ and $w_1(l_1) > w_1(l_2)$}
    This is analogous to \textbf{case 1}, and proceeds as mentioned above.
    \case{$w_0(l_1) < w_0(l_2)$ and $w_1(l_1) < w_1(l_2)$}
    One cycle, $l_2$, increases both Player 0 and Player 1's mean-payoffs, while the other, $l_1$, decreases it. In this case, we can just omit one of the cycles - whichever cycle gives a larger mean-payoff, to get a finite memory strategy. Thus, $\pi' = \pi_1.l_2^{\omega}$ and we get $\underline{\mathbf{MP}}_0(\pi') > c$ , $\underline{\mathbf{MP}}_1(\pi') \geqslant d$.
    \case{$w_0(l_1) > w_0(l_2)$ and $w_1(l_1) > w_1(l_2)$}
    This is analogous to \textbf{case 3} and proceeds as mentioned above.
\end{caseof}
In each of these cases, we can show that $\pi'$ is a witness for $\mathbf{ASV}^{\epsilon}(v) > c$: $\underline{\mathbf{MP}}_0(\pi') > c$ and $\underline{\mathbf{MP}}_1(\pi') \geqslant d$.
Since we know that $\pi$ does not cross a $(c,d)^{\epsilon}$-bad vertex, the modified play $\pi'$ does not either, as the vertices of the play $\pi'$ are a subset of the vertices of the play $\pi$. 

For the left to right direction of the proof, we are given $\mathbf{ASV}^{\epsilon}(v) > c$. We can construct a path $\pi$ from \textbf{\cref{ThmWitnessASVInfMem}} such that $\underline{\mathbf{MP}}_0(\pi) > c$ and $\underline{\mathbf{MP}}_1(\pi) = d$ and $\pi$ does not cross a $(c,d)^{\epsilon}$-bad vertex, i.e., for all vertices $v'$ appearing in $\pi$, we have that $v' \nvDash \ll 1 \gg \underline{\mathbf{MP}}_0(\pi) \leqslant c \land \underline{\mathbf{MP}}_1(\pi) > d-\epsilon$. Let $\inf(\pi) = S$ be the set of vertices appearing infinitely often in $\pi$. Note that $S$ forms an SCC. 
Following the proof in \textbf{Lemma 10} in \cite{FGR20} and applying the Carathéodory baricenter theorem, we can find two simple cycles $l_1, l_2$ in $S$ and acyclic finite plays $\pi_1, \pi_2 $ and $\pi_3$ from $\pi$ and two positive rational constants $\alpha, \beta \in \mathbb{Q^+}$, such that, \textit{first}($\pi_1$) $ = v$, \textit{first}($\pi_2$) = \textit{last}($\pi_1$), \textit{first}($\pi_3$) = \textit{last}($\pi_2$), \textit{first}($\pi_2$) = \textit{last}($\pi_3$), and \textit{first}($\pi_2$) = \textit{first}($l_1$), and \textit{first}($\pi_3$) = \textit{first}($l_2$), and $\alpha + \beta = 1$, $\alpha \cdot w_0(l_1) + \beta \cdot w_0(l_2) > c$ and $\alpha \cdot w_1(l_1) + \beta \cdot w_1(l_2) = d' \geqslant d$. We note that for all vertices $v$ in $\pi_1, \pi_2, \pi_3, l_1$ and $l_2$, we have that $v$ is not $(c,d)^{\epsilon}$-bad and thus will not be $(c,d')^{\epsilon}$-bad.
\end{proof}

Now, we state the following theorem that establishes the $\mathbf{NP}$-membership of $\mathbf{ASV}^{\epsilon}(v) > c$.
\begin{theorem}
    \label{ThmNpForASV}
    Given a mean-payoff game $\mathcal{G}$, a vertex $v \in V$, a rational value $c \in \mathbb{Q}$ and $\epsilon > 0$, it can be decided in non-deterministic polynomial time if $\mathbf{ASV}^{\epsilon}(v) > c$ and a finite memory strategy of Player~0 suffices for this threshold.
\end{theorem}

\noindent The rest of this \mychapter is devoted towards proving \textbf{\cref{ThmNpForASV}}. We start by stating a property of multi-dimensional mean-payoff games proved in \cite{VCDHRR15} that we rephrase here for a 2-dimensional game. This property expresses a bound on the weight of every finite play $\pi^f \in \mathbf{Out}_v(\sigma_0)$ where $\sigma_0$ is a memoryless winning strategy for Player~0.

\begin{lemma}
    \label{LemWeightPlayGrtThanC}
    In a mean-payoff game $\mathcal{G}$, if Player~0 wins $\underline{\mathbf{MP}}_0 < c \lor \underline{\mathbf{MP}}_1 < d$ from a vertex $v$ then he has a memoryless winning strategy $\sigma_0$ to do so, and there exist three constants $m_\mathcal{G}, c_\mathcal{G}, d_\mathcal{G} \in \mathbb{R}$ such that: $c_\mathcal{G} < c, d_\mathcal{G} < d$, and for all finite plays $\pi^f \in \mathbf{Out}_v(\sigma_0)$, i.e. starting in $v$ and compatible with $\sigma_0$, we have that:
    \begin{equation*}
        w_0(\pi^f) \leqslant m_\mathcal{G} + c_\mathcal{G} \times |\pi^f|
    \end{equation*}
    or
    \begin{equation*}
        w_1(\pi^f) \leqslant m_\mathcal{G} + d_\mathcal{G} \times |\pi^f|
    \end{equation*}
\end{lemma}

\noindent If Player~0 is to ensure that Player~1 does not deviate from a certain path $\pi$, Player~0 must ensure that from every vertex $v$ in $\pi$, $v \nvDash \ll 1 \gg \underline{\mathbf{MP}}_0 \leqslant c \land \underline{\mathbf{MP}}_1 > d-\epsilon$. Using \textbf{\Cref{LemWeightPlayGrtThanC}}, we can now prove if Player~0 can ensure from a vertex $v$ that $v \nvDash \ll 1 \gg \underline{\mathbf{MP}}_0 \leqslant c \land \underline{\mathbf{MP}}_1 > d-\epsilon$, then he can also ensure that $v \nvDash \ll 1 \gg \underline{\mathbf{MP}}_0 \leqslant c \land \underline{\mathbf{MP}}_1 \geqslant d-\epsilon$. This is established in the following result.

\begin{lemma}
    \label{ConjGrtIsGrtEq}
    For every mean payoff game $\mathcal{G}$, for all vertices $v \in \mathcal{G}$, for all rational constants $c, d \in \mathbb{Q}$, we have that:
    \begin{equation*}
        v \models \ll 1 \gg \underline{\mathbf{MP}}_0 \geqslant c \land \underline{\mathbf{MP}}_1 > d
    \end{equation*}
    if and only if there exists a rational constant $d' \in \mathbb{Q}$, where $d' > d$ such that
    \begin{equation*}
        v \models \ll 1 \gg \underline{\mathbf{MP}}_0 \geqslant c \land \underline{\mathbf{MP}}_1 \geqslant d'
    \end{equation*}
\end{lemma}

\begin{proof}
    For the right to left direction of the proof, it is trivial to see that 
    % if $\underline{\mathbf{MP}}_1 > d'$ for some $d' > d$ then this implies $\underline{\mathbf{MP}}_1 \geqslant d$. Hence we get that 
    if $v \models \ll 1 \gg \underline{\mathbf{MP}}_0 \geqslant c \land \underline{\mathbf{MP}}_1 \geqslant d'$ for some $d' > d$, then we have that $v \models \ll 1 \gg \underline{\mathbf{MP}}_0 \geqslant c \land \underline{\mathbf{MP}}_1 > d$.

    For the left to right direction of the proof, we prove the contrapositive, i.e., we assume that $\forall d' > d$, we have $v \nvDash \ll 1 \gg \underline{\mathbf{MP}}_0 \geqslant c \land \underline{\mathbf{MP}}_1 \geqslant d'$. Now we prove that $v \nvDash \ll 1 \gg \underline{\mathbf{MP}}_0 \geqslant c \land \underline{\mathbf{MP}}_1 > d$.

    Since $\forall d'> d$, Player~1 loses $\underline{\mathbf{MP}}_0 \geqslant c \land \underline{\mathbf{MP}}_1 \geqslant d'$ from a given vertex $v$, due to determinacy of multi-dimensional mean-payoff games, Player~0 wins $\underline{\mathbf{MP}}_0 < c \lor \underline{\mathbf{MP}}_1 < d'$ from the same vertex $v$. By \textbf{\Cref{LemWeightPlayGrtThanC}}, we have that Player~0 has a memoryless strategy $\sigma_0$ to achieve the objective $\underline{\mathbf{MP}}_0 < c \lor \underline{\mathbf{MP}}_1 < d'$ from the vertex $v$. Note that in a finite game $\mathcal{G}$, Player~0 has only finitely many memoryless strategy. Therefore there must exist at least one strategy $\sigma_0^*$ that achieves the objective $v \models \ll 0 \gg \underline{\mathbf{MP}}_0 < c \lor \underline{\mathbf{MP}}_1 < d'$ for all $d' > d$. From \textbf{\Cref{LemWeightPlayGrtThanC}}, we also get three constants $m_\mathcal{G}, c_\mathcal{G}, d_\mathcal{G} \in \mathbb{R}$ such that for all $d' > d$, we have that $c_\mathcal{G} < c, d_\mathcal{G} < d'$ and for all finite plays $\pi^f \in \mathbf{Out}_v(\sigma_0^*)$, we have that
    \begin{equation*}
        w_0(\pi^f) \leqslant m_\mathcal{G} + c_\mathcal{G} \times |\pi^f|
    \end{equation*}
    or we have that
    \begin{equation*}
        w_1(\pi^f) \leqslant m_\mathcal{G} + d_\mathcal{G} \times |\pi^f|
    \end{equation*}

    Now, for every play $\pi \in \mathbf{Out}_v(\sigma_0^*)$, we know that $|\pi| \to \infty$. Thus, we get that:
    \begin{equation*}
        \underline{\mathbf{MP}}_0(\pi) = \frac{w_0(\pi)}{|\pi|} \leqslant \frac{m_\mathcal{G}}{|\pi|} + c_\mathcal{G}
    \end{equation*}
    or we have that
    \begin{equation*}
        \underline{\mathbf{MP}}_1(\pi) = \frac{w_1(\pi)}{|\pi|} \leqslant \frac{m_\mathcal{G}}{|\pi|} + d_\mathcal{G}
    \end{equation*}

    Since $|\pi| \to \infty$, the constant $\frac{m_\mathcal{G}}{|\pi|} \to 0$. Thus, we get $v \models \ll 0 \gg \underline{\mathbf{MP}}_0(\pi) \leqslant c_\mathcal{G} \lor \underline{\mathbf{MP}}_1(\pi) \leqslant d_\mathcal{G}$. Since we know that $c_\mathcal{G} < c, d_\mathcal{G} < d'$, we get that $v \models \ll 0 \gg \underline{\mathbf{MP}}_0(\pi) < c \lor \underline{\mathbf{MP}}_1(\pi) < d'$ for every $d' > d$. Thus, we get that  $v \models \ll 0 \gg \underline{\mathbf{MP}}_0(\pi) < c \lor \underline{\mathbf{MP}}_1(\pi) \leqslant d$. Therefore, by determinacy of multi-dimensional mean-payoff games, we have that $v \nvDash \ll 1 \gg \underline{\mathbf{MP}}_0 \geqslant c \land \underline{\mathbf{MP}}_1 > d$.
\end{proof}

\textbf{Modified Game} We now construct a game $\mathcal{G'}$ from the given game $\mathcal{G}$ by multiplying the first dimension on each edge by -1. It is easy to see that, for all vertices $v$ in $\mathcal{G}$, we have that $v \nvDash \ll 1 \gg \underline{\mathbf{MP}}_0 \leqslant c \land \underline{\mathbf{MP}}_1 > d-\epsilon$ if and only if for the same vertex $v$ in the game $\mathcal{G'}$, we have $v \nvDash \ll 1 \gg \overline{\mathbf{MP}}_0 \geqslant -c \land \underline{\mathbf{MP}}_1 > d-\epsilon$ for some value $d \in \mathbb{Q}$. From \textbf{\cref{ConjGrtIsGrtEq}}, for the same vertex $v$, we have $v \nvDash \ll 1 \gg \overline{\mathbf{MP}}_0 \geqslant -c \land \underline{\mathbf{MP}}_1 \geqslant d'$, for some $d' \in \mathbb{R}$ and $d' > d - \epsilon$. \\ \noindent
We modify the game $\mathcal{G'}$ further by adding $c$ to the first dimension and subtracting $d'$ from the second dimension for all the edges in the game. Note that the edges and the vertices of both games $\mathcal{G}$ and $\mathcal{G'}$ are the same, the only difference between the two games being their edge weights.

\noindent The following proposition states that the set of winning strategies for both Player~0 and Player~1 are the same in the games $\mathcal{G}$ and $\mathcal{G'}$.

\begin{proposition}
    \label{PropGamStrEqNewGamStr}
    For every vertex $v \in V$ in the game $\mathcal{G'}$, Player~0 (Player~1) can ensure that $v \nvDash \ll 1 \gg \overline{\mathbf{MP}}_0 \geqslant 0 \land \underline{\mathbf{MP}}_1 \geqslant 0$ ($v \models \ll 1 \gg \overline{\mathbf{MP}}_0 \geqslant 0 \land \underline{\mathbf{MP}}_1 \geqslant 0$) with a strategy $\sigma_0^{v}$ ($\sigma_1^{v}$) if and only if Player~0 (Player~1) can ensure that $v \nvDash \ll 1 \gg \underline{\mathbf{MP}}_0 \leqslant c \land \underline{\mathbf{MP}}_1 > d-\epsilon$ ($v \models \ll 1 \gg \underline{\mathbf{MP}}_0 \leqslant c \land \underline{\mathbf{MP}}_1 > d-\epsilon$) with the same strategy $\sigma_0^{v}$ ($\sigma_1^{v}$) from vertex $v$ in the game $\mathcal{G}$.
\end{proposition}

\noindent We also refer to the following result for Multi-weighted two-player games with $\mathbf{MeanPayoffInfSup}$ objective \cite{VCDHRR15}. This result establishes that if Player~0 can ensure that Player~1 does not satisfy the $\mathbf{MeanPayoffInfSup}$ objective from a vertex $v$, then she has a memoryless strategy for doing so.

\begin{theorem}
    \label{ThmMemlessStrForP2}
    \textbf{\emph{(Theorem 8 in \cite{VCDHRR15})}} For multi-dimensional two-player mean-payoff games with objective \\
    $\mathbf{MeanPayoffInfSup}(I,J) = \{\pi' \in \mathbf{Plays}(\mathcal{G}) \mid \forall i \in I : \underline{\mathbf{MP}}(\pi')_i \geqslant 0 \text{ and } \forall j \in J : \overline{\mathbf{MP}}(\pi')_j \geqslant 0\}$ for Player~1, the following assertions hold:
    \begin{enumerate}
        \item Winning strategies for Player~1 require infinite-memory in general, and memoryless winning strategies exist for Player~0. \footnote{Player~0 is called Player 2 in \cite{FGR20}}
        \item The problem of deciding whether a given vertex is winning for Player~1 is coNP-complete.
    \end{enumerate}
\end{theorem}
 
\begin{proof}[Proof of \textbf{\Cref{ThmNpForASV}}]
According to the \textbf{\cref{LemPlaysAsWitnessForASV}}, we consider a non-deterministic Turing machine that establishes the membership to $\mathbf{NP}$ by guessing a reachable SCC $S$, a finite play $\pi_1$ to reach $S$ from $v$, two simple cycles $l_1, l_2$, along with two finite plays $\pi_2$ and $\pi_3$ that connects the two simple cycles, and parameters $\alpha, \beta \in \mathbb{Q}^{+}$. Note that the parameters $\alpha , \beta$ can be obtained by solving a linear program. Since linear programs are solvable in polynomial time, the values $\alpha$ and $\beta$ have polynomial size representation. Additionally, for each vertex $v'$ that appear along the plays $\pi_1, \pi_2$ and $\pi_3$, and on the simple cycles $l_1$ and $l_2$, the turing machine guesses a memoryless strategy $\sigma_0^{v'}$ for Player~0 that establishes $v' \nvDash \ll 1 \gg \underline{\textbf{MP}}_0 \leqslant c \land \underline{\textbf{MP}}_1 > d - \epsilon$ which means by determinacy of multi-dimensional mean-payoff games, that $v' \models \ll 0 \gg \underline{\textbf{MP}}_0 > c \lor \underline{\textbf{MP}}_1 \leqslant d - \epsilon$.

Addtionally, note that from \textbf{\cref{LemPlaysAsWitnessForASV}}, we can obtain a $\epsilon$-regular-witness $\pi'$. Using $\pi'$, we build a finite memory strategy $\sigma_0^v$ for Player~0 as stated below:
\begin{enumerate}
    \item Player~0 follows $\pi'$ if Player~1 does not deviate from $\pi'$. The finite memory strategy stems from the finite $k$ as required in the four cases mentioned above.
    \item For each vertex $v \in \pi'$, Player~0 employs a memoryless strategy that establishes $v \nvDash \ll 1 \gg \underline{\mathbf{MP}}_0 \leqslant c \land \underline{\mathbf{MP}}_1 > d-\epsilon$.
\end{enumerate}

We shall now establish the existence of these memoryless strategies. 
We start by construction of a modified mean-payoff game $\mathcal{G'}$ from the given game $\mathcal{G}$. Now, we get a modified objective for Player~0 in $\mathcal{G'}$, i.e., to get a memoryless strategy for Player~0 for all vertices $v \in \pi_1, \pi_2, \pi_3, l_1, l_2$ such that $v \nvDash \ll 1 \gg \overline{\mathbf{MP}}_0 \geqslant 0 \land \underline{\mathbf{MP}}_1 \geqslant 0$. The existence of a memoryless strategy for Player~0 which ensures her objective follows from \textbf{\Cref{ThmMemlessStrForP2}}

\noindent We formulate $\mathcal{G'}$ as a multi-weighted two-player game structure specified in \textbf{\cref{ThmMemlessStrForP2}} with $I = \{1\}$ and $J = \{0\}$.
Thus, by \textbf{\cref{PropGamStrEqNewGamStr}}, if Player~0 has a strategy $\sigma_0^{v}$ to ensure $v \nvDash \ll 1 \gg \underline{\mathbf{MP}}_0 \leqslant c \land \underline{\mathbf{MP}}_1 > d-\epsilon$ from every vertex $v \in \pi_1, \pi_2, \pi_3, l_1, l_2$, the same strategy $\sigma_0^{v}$for every vertex $v$ appearing in $\pi_1, \pi_2, \pi_3, l_1, \text{ and } l_2$ in the game $\mathcal{G'}$ ensures $v \nvDash \ll 1 \gg \underline{\mathbf{MP}}_0 \geqslant 0 \land \underline{\mathbf{MP}}_1 \geqslant 0$. 
By \textbf{\cref{ThmMemlessStrForP2}}, since Player~0 wins the game $\mathcal{G}'$ from every vertex $v$ in $\pi_1, \pi_2, \pi_3, l_1, s \text{ and } l_2$, she has a memoryless strategy $\sigma_0^{v}$ to do so thus ensuring that $v \nvDash \ll 1 \gg \underline{\mathbf{MP}}_0 \leqslant c \land \underline{\mathbf{MP}}_1 > d - \epsilon$. 
There are polynomially many vertices, and the memoryless strategy $\sigma_0^{v}$ is checkable in $\mathsf{PTime}$ ensuring that the vertices along $\pi_1, \pi_2, \pi_3$ and $l_1, l_2$ are not $(c,d)^\epsilon$-bad, where $d = \alpha \cdot w_1(l_1) + \beta \cdot w_1(l_2)$.
\end{proof}